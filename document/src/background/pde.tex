Partial differential equations are equations of an unknown multivariable function and its partial derivatives.
There are a large family of different PDEs serving different purposes from pure mathematics to physics, and finance.
They are especially useful in modeling different kinds of systems, which is also the case of this thesis.

In modeling spatiotemporal data, the advection-diffusion equation is an often used generalisation of a system.
The advection-diffusion equation is defined in space $\Omega \subset \R^n$, and time $I = (0, T)$ as

\begin{equation}\label{eq:convdif}
    \frac{\partial u}{\partial t} = \nabla \cdot (D \nabla u) - \mathbf{F} \nabla u.
\end{equation}
The gradients are taken here with respect to the spatial variable.
Function $u = u(x,t)$ represents the quantity of interest in $\Omega$ at time $t$.
$D = D(x,t)$ is the diffusion coefficient function, and $\mathbf{F} = \mathbf{F}(x,t)$ the vector-valued advection field that transports $u$ forward in time.
In many practical uses $D$ can be constant in space, and then the diffusion term $\nabla \cdot (D \nabla u)$ simplifies to $D \Delta u$.

Here if the diffusion coefficient $D$ is set to zero the equation reduces to the advection equation where objects are just transported through $\Omega$ with the vector field $\mathbf{F}$.
Then the equation could be thought to represent an optical flow problem.

If the vector field $\mathbf{F}$ is pushed to zero, the equation is just a diffusion equation, and in the case where $D$ is constant, the PDE becomes the heat equation.
In that case objects would only diffuse in time without moving anywhere.

An example of a process that can be thought to adhere according to the equation some sort of cloudiness data.
$u$ can be for example two-dimensional data collected by satellites on cloudiness.
But as it turns out, clouds don't diffuse in the atmosphere, and thus the diffusion term could be modelled to be very small and can be thought to represent some kind of uncertainty when forecasting.

The typical forward problem of the PDE~\eqref{eq:convdif} would be the to solve $u$ given some initial condition
\begin{equation}
    u(\cdot, 0) = u_0(\cdot),
\end{equation}
and a boundary condition for the values of $u$ in the Dirichlet type case or the normal derivative of $u$ along the boundary of $\Omega$ in the Neumann type case.
However, solving the forward problem would require knowledge of the advection field $\mathbf{F}$, and the diffusion coefficient $D$.

In practice, with for example the cloud thickness case measuring $\mathbf{F}$ and $D$ is impossible, but satellites collect constantly remote sensing data from $u$ which leads one to the inverse problem.
An inverse problem posed by the advection-diffusion equation~\eqref{eq:convdif} would be to solve $D$ and $\mathbf{F}$ given some measurements from $u$.
The measurements of $u$ could be taken from the boundary for a boundary value inverse problem or in the cloud example measurements are taken from the entire $\Omega$.
One motivation behind solving the inverse problem is that assuming the solution changes slowly in time, the parameters could then be used to move $u$ forward in time to produce short-term forecasts.

