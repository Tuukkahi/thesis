Neural networks have proved themselves an important tool in the analysis of high-dimensional and complex data being able to learn patterns and relationships.
In meteorological nowcasting, the so-called physics-free models have been extensively researched in e.g.\ percipitation forecasts.~\cite{rainnet, metnet}
While these methods have shown some efficiency in producing realistic forecasts, in many cases it would be useful to constrain these methods to adhere to some physical models.

Bayesian filtering techniques including the well-known Kalman filter provide a cruical role in state estimation of dynamic systems.
The use of neural networks in parameter estimation could offer a powerful method of parameter estimation in the Bayesian filtering framework involving advective spatio-temporal data.

This thesis aims on devoloping a neural network for advection field estimation from noisy spatio-temporal data.
The advection fields are used to model a state-space model in the Bayesian filtering framework.
As a practical application, the model is studied on a cloud optical thickness data where the model is used to get probabilistic short-term forecasts.

The main contribution of the thesis is to integration of a neural network model to a high-dimensional Kalman filtering scheme.
In addition, the physics-based neural network model is tested against classical methods used in advective spatio-temporal meteorological forecasting.

The thesis unfolds as follows.
First, the needed background material to neural networks and Bayesian filtering of spatio-temporal data are presented.
Next, the theoretical properties of the proposed model is covered, and in the final chapter the practical application in cloud optical thickness forecasting is presented.
