Bayesian filtering refers to to the Bayesian way of deriving an optimal filtering problem of estimating the state of a nonstationary system through some observations.
This approach becomes valuable when dealing with real-world phenomena where a mathematical model can approximate the temporal evolution of some process and measurements of the system may be uncertain.


A key application of Bayesian filtering is state estimation which emerges in many disciplines.
It involves estimating a hidden state of a system based on some available indirect and noisy measurements.
Bayesian filtering provides a robust mathematical framework to combine prior knowledge about a system's state with new measurements to enhance state estimates over time.

The Bayesian filtering theory emerged at the turn of the 1960s with perhaps the most famous example being the works of Rudolf E. Kálmán who published a recursive solution to a linear special case of the problem in~\cite{kalman}.
Meanwhile in the Soviet Union Ruslan, Stratonovich studied the more general nonlinear Bayesian filtering problem, see e.g.~\cite{stratonovich2, stratonovich1}.
The simultaneous advances in digital computers allowed the theory to prove itself quickly useful in many applications, an early well-known example of which is the guidance system of the Apollo spacecraft~\cite{mcgee}.

In the context of this work, Bayesian filtering is employed to manage uncertainties related to the neural network model.
The neural network is capable of capturing the advection fields, and the Bayesian filtering framework provides a way to generate probailistic state estimates of the COT data.

\subsubsection{Bayesian Inference}
Bayesian inference refers to the process of updating prior probability estimates in light of more information.
The basis of Bayesian inference is the Bayes' Theorem.
\begin{theorem}[Bayes' Theorem]
    Assume a known prior probability density function $\pi(x)$ from a random variable $X$, and an observed probability $\pi(y)$ from another random variable $Y$.
    Then the posterior distribution of $X$ given the observed data is given by
    \begin{equation}
        \pi(x \, \vert \, y) = \frac{\pi(x) \pi(y \, \vert \, x)}{\pi(y)}.
    \end{equation}
\end{theorem}
In Bayesian inference, the prior distribution contains the information of some quantity before observing the data, and the posterior distribution the updated information after the observation.
Bayesian inference revolves around using data to transform the prior distributions into the posterior distributions.

\subsubsection{Probabilistic state-space models}
The Bayesian framework of assimilating measurements, using a process model, and producing forecasts from them can be formalised as the state-space model.
Given the inherent uncertainties linked with measurments and modelling, instead of focusing solely on point estimates of the state, it is substantially more fruitful to consider probability distributions.

Consider two stochastic processes, $\{X_k\}_{k=0}^{\infty}$ and $\{Y_k\}_{k=0}^{\infty}$, denoting the state and measurements of a system respectively.
Figure~\ref{fig:statespacedependency} presents the dependency scheme of the processes needed to formulate the state-space model.
\begin{figure}[h]
    \begin{alignat*}{5}
    \text{State:}& \quad && X_0 \, \rightarrow \, && X_1 \, \rightarrow \, && X_2 \,\rightarrow \cdots \rightarrow \, && X_k \, \rightarrow \cdots \\
                 & &&  && \downarrow && \downarrow && \downarrow \\
    \text{Observations:}& \quad && && Y_1 && Y_2 && Y_k
\end{alignat*}

    \caption{\label{fig:statespacedependency}Dependency scheme of the state-space model.}
\end{figure}
In order for the dependencies of the figure to materialise, some assumptions about the processes are required:
\begin{enumerate}
    \item 
        $\{X_k\}_{k=0}^{\infty}$ satisfies the Markov property with respect to its own history, and history of the observations:
        \begin{equation*}
            \pi (x_{k+1} \, \vert \, x_{0:k}, y_{1:k}) = \pi (x_{k+1} \, \vert \, x_k).
        \end{equation*}
    \item
        $\{Y_k\}_{k=0}^{\infty}$ satisfies the Markov property with respect to the history of the true state:
        \begin{equation*}
            \pi (y_k \, \vert \, x_{0:k}) = \pi (y_k \, \vert \, x_k).
        \end{equation*}
\end{enumerate}
With these assumptions the probabilistic state-space model can be defined.
\begin{definition}[Probabilstic state-space model]\label{def:state-space}
    Assume stochastic processes $\{X_k\}_{k=0}^{\infty}$ and $\{Y_k\}_{k=0}^{\infty}$ meet the previous assumptions.
    Then the following conditional probability distributions define the probabilstic state-space model:
    \begin{subequations}\label{eq:statespace}
        \begin{align}
            X_0 &\sim \pi (x_0)\\
            X_{k+1} &\sim \pi (x_{k+1} \, \vert \, x_{k}), \quad k = 0,1,2,\dots \label{eq:statespace1}\\
            Y_k &\sim \pi (y_k \, \vert \, x_k), \quad k = 1,2,\dots. \label{eq:statespace2}
        \end{align}
    \end{subequations}
\end{definition}
In Definition~\ref{def:state-space},~\eqref{eq:statespace1} is the model that describes the evolution of state over time, and~\eqref{eq:statespace2} is the observation model that describes the distribution of measurements conditioned on the state.

If functions for the state evolution and observations are known, the same can also be alternatively written as 
\begin{subequations}\label{eq:statespacemodel}
    \begin{align}
        X_{k+1} &= M_{k+1}(X_{k}, W_{k+1}), \quad k = 0,1,\dots \label{eq:statespacemodel1}\\
        Y_{k} &= H_{k}(X_{k}, V_{k}), \quad k = 1,2,\dots. \label{eq:statespacemodel2}
    \end{align}
\end{subequations}
Now~\eqref{eq:statespacemodel1} is the state evolution equation where function $M_{k+1}$ gives the next state $X_{k+1}$ given the previous state $X_{k}$ and the state noise $W_{k+1}$
The observation equation~\eqref{eq:statespacemodel2} describes the observations $Y_{k}$ as being the output of the observation function $H_{k}$ with the true state $X_{k}$ and the observation noise $V_{k}$.

For the probailistic state-space models, a few conditional probability distributions are usually of interest outlined in the following definition.
\begin{definition}
    Assume two stochastic processes $\{X_k\}_{k=0}^{\infty}$ and $\{Y_k\}_{k=0}^{\infty}$ form a probabilistic state-space model.
    Then the conditional probability distribution
    \begin{itemize}
        \item
            $\pi(x_k \, \vert \, y_{1:k})$ is a filtering distribution.
        \item
            $\pi(x_{k+n}\, \vert \, y_{1:k}), \, n > 0$ is a prediction distribution.
        \item
            $\pi(x_k \, \vert \, y_{1:})$ is a smoothing distribution.
    \end{itemize}
\end{definition}

With the state-space model~\eqref{eq:statespace} the Bayesian filtering problems can be discussed.
The idea behind Bayesian filtering is to recursively update the predictions and states of the model given latest observations.
The following theorem presents the two update equations for the Bayesian filtering scheme.
\begin{theorem}[Bayesian filtering equations]\label{thm:updateformulas}
    Let $\{X_k\}_{k=0}^{\infty}$ be a state process and $\{Y_k\}_{k=0}^{\infty}$ an observation process in a state-space model.
    Then the following recursive equations for Bayesian filtering steps apply:
    \begin{enumerate}
        \item 
            Prediction step:
            \begin{equation}\label{eq:predictionstep}
                \pi (x_{k+1} \, \vert \, y_{1:k}) = \int \pi (x_{k+1} \, \vert \, x_k) \pi (x_k \, \vert \, y_{1:k}) \, d x_k.
            \end{equation}
        \item
            Observation update step:
            \begin{equation}\label{eq:updatestep}
                \pi(x_{k+1} \, \vert \, y_{1:k+1}) = \frac{\pi (y_{k+1} \, \vert \, x_{k+1}) \pi (x_{k+1} | y_{1:k})}{\pi (y_{k+1} \, \vert \, y_{1:k})},
            \end{equation}
            with
            \begin{equation*}
                \pi (y_{k+1} \, \vert \, y_{1:k}) = \int \pi (y_k \, \vert \, x_k) \pi (x_k \, \vert \, y_{1:k-1}) \, d x_{k+1}.
            \end{equation*}
    \end{enumerate}
\end{theorem}
\begin{proof}
    In Appendix~\ref{app:proofs}.
\end{proof}

Theorem~\ref{thm:updateformulas} provides a framwork for efficiently computing probability distributions conditioned on observations.
The prediction step equation~\ref{eq:predictionstep} being the joint probability distribution of the current and next state conditioned on the latest observations facilitates probabilistic state estimates.
Then the observation update equation~\ref{eq:updatestep} is just the Bayes formula where $\pi (x_{k+1} \, \vert \, y_{1:k})$ is interpreted as the prior information of the state for statistical inversion of a new observation.

\subsubsection{Kalman Filter}
The Kalman filter provides an elegant closed-form solution for the update equations of Theorem~\ref{thm:updateformulas} in the special case of a linear state-space model with Gaussian additive noise process.
In this special case the state-space model corresponding to~\eqref{eq:statespacemodel} becomes
\begin{subequations}\label{eq:kalmanstatespace}
    \begin{align}
        X_{k+1} &= M_{k+1}X_{k}  + W_{k+1}, \quad W_{k+1} \sim \mathcal{N}(0, Q_{k+1})\\
        Y_{k} &= H_{k}X_{k} + V_{k}, \quad V_t \sim \mathcal{N}(0, R_k)
    \end{align}
\end{subequations}
with a Gaussian distribution for the initial state $X_0$.

Under these assumptions, the state's probability densities of the state of the model can be given by just estimates of the mean and covariance.
The following theorem gives the filtering step equations of Theorem~\ref{thm:updateformulas}.
\begin{theorem}[Kalman filter equations]\label{thm:kalmanfilterupdate}
    Assume a linear Gaussian state-space model~\eqref{eq:kalmanstatespace}.
    Let $x_{k\,\vert\,\ell}$ and $P_{k\,\vert\,\ell}$ represent the a posteriori mean and covariance of the state at time $k$ given observations $y_{1:\ell}$.
    Then the filterering equations of Theorem~\ref{thm:updateformulas} take closed form solutions:
    \begin{enumerate}
        \item 
            Prediction step:
            Given
            \begin{equation*}
                \pi (x_k \, \vert \, y_{1:k}) \sim \mathcal{N}(x_{k\,\vert\,k}, P_{k\,\vert\,k}),
            \end{equation*}
            a priori distribution of the next state is
            \begin{equation*}
                \pi (x_{k+1} \, \vert \,  y_{1:k}) \sim \mathcal{N}(\hat{x}_{k+1\,\vert\,k}, \hat{P}_{k+1\,\vert\,k}),
            \end{equation*}
            where parameters of the distribution are given by
            \begin{subequations}\label{eq:kalmanpredictionstep}
                \begin{align}
                    \hat{x}_{k+1\,\vert\,k} &= M_{k+1}x_{k\,\vert\,k}\\
                    \hat{P}_{k+1\,\vert\,k} &= M_{k+1}P_{k\,\vert\,k}M_{k+1}^{\top} + Q_{k+1}.
                \end{align}
            \end{subequations}
        \item
            Observation update step:
            Given
            \begin{equation*}
                \pi (x_{k+1} \, \vert \,  y_{1:k}) \sim \mathcal{N}(x_{k+1\,\vert\,k}, P_{k+1\,\vert\,k}),
            \end{equation*}
            a posteriori distribution of the next state is
            \begin{equation*}
                \pi (x_{k+1} \, \vert \,  y_{1:k+1}) \sim \mathcal{N}(x_{k+1\,\vert\,k+1}, P_{k+1\,\vert\,k+1}).
            \end{equation*}
            With 
            \begin{subequations}\label{eq:kalmanupdatestep}
                \begin{align}
                    v_{k+1} &= y_{k+1} - H_{k+1}x_{k+1 \, \vert \, k},\\
                    S_{k+1} &= H_{k+1} \hat{P}_{k+1\,\vert\,k} H_{k+1}^{\top},\\
                    K_{k+1} &= \hat{P}_{k+1\,\vert\,k}  H_{k+1}^{\top} S_{k+1}^{-1},\\
                    x_{k+1\,\vert\,k+1} &= \hat{x}_{k+1\,\vert\,k} + K_{k+1} v_{k+1},\\
                    P_{k+1\,\vert\,k+1} &= (I-K_{k+1}H_{k+1})\hat{P}_{k+1\,\vert\,k}.
                \end{align}
            \end{subequations}
    \end{enumerate}
\end{theorem}
\begin{proof}
    In Appendix~\ref{app:proofs}.
\end{proof}
While the theorem assumes zero means for noises, the Gaussian distributions can easiliy be shifted to accomodate for non-zero means, too.
The Kalman filter leverages the property of Gaussian distributions retaining their Gaussian form under affine transformations.
However, for nonlinear state-space models one can use linear approximations in order to preserve the convenience of Gaussian densities, a method known as extended Kalman filter.

Even when the assumptions for the Kalman filter apply, with larger states and observation spaces, problems will arise with computing the Kalman filter steps.
The matrix update formulas of Theorem~\ref{thm:kalmanfilterupdate} cross quickly the limits of computational feasibility.
One way of tackling the problem is to use Ensemble Kalman filters.

\subsubsection{Ensemble Kalman Filter}
Despite the effectiveness of the original Kalman filter in optimal filtering problems, it doesn't scale well with large state and observation spaces.
This limitation leads to the Ensemble Kalman Filter (EnKF).
It was first introduced by Geir Evensen in~\cite{evensen1994} as a Monte Carlo approach to the Kalman filter.
EnKF circumvents the large covariance matrices by representing a state of an state-space model by Monte Carlo samples from which sample covariances are computed.
