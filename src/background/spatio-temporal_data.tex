The analysis of spatio-temporal data is an important part of research across many disciplines.
Spatio-temporal data is characterised by having both spatial, and temporal components that carry information across space, and over time.
Modelling these datasets requires thus observing and understanding how the phenomenon of interests evolves over time in many locations.
This subsection covers some basic principles of spatio-temporal data and dynamic modelling such data through the advection equation. 

\subsubsection{Defining Spatio-Temporal Data}
As to dismiss the Einstenian physics where space and time is interdependent, the classical way of separating the real spatial and temporal dimensions shall be considered here.

So now the spatial domain $\Omega$ is assumed to be a subset of $n$-dimensional real space, $\Omega \subset \R^n$, and the temporal domain $I$ a subset of the real line, $I \subset \R^1$.
Then the spatio-temporal domain is the Cartesian product of the spatial and temporal domains, $\Omega \times I$ and then an arbitrary spatio-temporal process can be written as
\begin{equation}
    \left\{ Y(x, t) \, \middle\vert \, x \in \Omega, \, t \in I \right\}.
\end{equation}

This way the spatio-temporal data may be thought as snapshots of some process at a specific point in time.
Naturally either, or both, the spatial and temporal variables may also be discrete by the nature of the quantities or due to discrete measurements.

A concrete example of spatio-temporal data could be air pollution data that tracks the concentration of pollutants accross different areas in a city and how it evolves over time.

\subsubsection{Dynamic Modelling of Spatio-Temporal Data}\label{subsubsection:dynamicmodellingspatiotemporaldata}
Dynamical modelling of spatio-temporal data refers to creating mathematical and computational models that explain changes of the observed variables in time.
Often some prior scientific information of the process is used to motivate the model.
This information is usually described with some partial differential equations (PDE).
At the core of the model of this thesis is the advection-diffusion equation describing the transportation and diffusion of the observed variable along some advection field in time.
In the air pollution example, the pollutants would be advected by winds, and diffused outwards of high concentrations.
A simplified example of data evolving through the advection-diffusion process is in Figure~\ref{fig:advection_diffusion}.

\begin{figure}[h]
    \centering
    \inputpypgf{background/fig/advectiondiffusion.pypgf}
    \caption{\label{fig:advection_diffusion}An example the of advection-diffusion process with two gaussian kernels.}
\end{figure}

The advection-diffusion equation consist of an advection part that describes the transportation, and diffusion that describes the diffusion process.
Consider a spatio-temporal function $u: \Omega \times I \rightarrow \R$ representing some quantity on spatial domain $\Omega \subset \R^n$, and temporal domain $I = (0, T)$.
The plain advection part is defined as
\begin{equation}\label{eq:advection}
    \frac{\partial u}{\partial t} =  - \mathbf{F} \nabla u,
\end{equation}
where $\mathbf{F} = \mathbf{F}(x,t)$ is the vector-valued advection field that transport $u$ forward in time.

Then the diffusion equation is
\begin{equation}\label{eq:diffusion}
    \frac{\partial u}{\partial t} = \nabla \cdot (D \nabla u),
\end{equation}
with $D = D(x,t)$ being the diffusion coefficient.
In many practical uses $D$ is assumed to be constant in space, and then the diffusion term $\nabla \cdot (D \nabla u)$ simplifies to $D \Delta u$ and diffusion equation~\eqref{eq:diffusion} recovers to the heat equation.

Now by combining~\eqref{eq:advection} and~\eqref{eq:diffusion} gives the advection-diffusion equation
\begin{equation}\label{eq:advectiondiffusion}
    \frac{\partial u}{\partial t} = \nabla \cdot (D \nabla u) - \mathbf{F} \nabla u.
\end{equation}

If the advection field $\mathbf{F}$, and the diffusion coefficient $D$ were known, and the model was perfect $u$ could be solved given some initial condition $u(\cdot, 0) = u_0(\cdot)$, and a boundary condition for the values of $u$ in the Dirichlet type case or the normal derivative of $u$ along the boundary of $\Omega$ in the Neumann type case.
However, measuring advection and diffusion parameters is often not possible and estimates $\overline{\mathbf{F}}$ and $\overline{D}$ need to be extracted e.g.\ from data.
While many uncertainties are involved in the parameter estimation, the model~\eqref{eq:advectiondiffusion} is also not a perfect representation so the approximate model
\begin{equation}\label{eq:advectiondiffusion_approx}
    \frac{\partial u}{\partial t} = \nabla \cdot (\overline{D} \nabla u) - \overline{\mathbf{F}} \nabla u + e
\end{equation}
needs to be considered where $e = e(x,t)$ is the (hopefully small) modeling error.
When taking real measurements, the model also needs to be discretised both in time so that each time step represents one measurement point, and in space so that $\Omega$ becomes a grid in which the measurements are taken at some resolution.

In the air pollution example the motivation for computing the estimates $\overline{\mathbf{F}}$ and $\overline{D}$ would be that assuming the real parameters change little in time, i.e.\ $\mathbf{F}(x,t_{k}) \approx \mathbf{F}(x,t_{k+1})$ and $D(x,t_{k}) \approx D(x,t_{k+1})$, the parameters can be utilised to transport air pollution images $u$ to to future and thus producing forecasts.
